%
% Conclusion
%
This paper estimates how a key sector---agriculture---responds to changes in the cost of inputs---electricity and groundwater. We study the California agricultural sector, which produces 17 percent of total U.S.\ crop value. Because California farmers depend on irrigation, water scarcity has the potential to threaten farm production and profitability. This threat is only exacerbated by climate change, which is projected to increase the frequency and severity of droughts. 

To understand how California farms respond to rising water costs, we overcome previous measurement challenges with a novel dataset of restricted-access data on farmers' electricity consumption and groundwater pump efficiencies. We combined these data with government measurements of groundwater depths and satellite-derived land use designations for the universe of farmers in the Pacific Gas and Electric utility service territory, which covers the majority of the farmland in the state. We leverage exogenous variation in electricity tariffs over time to estimate farmers' price elasticity of demand for electricity, and find a surprising large elasticity estimate of $-1.17$. Then, we use physics to compute groundwater costs and groundwater quantities for each pump in our sample, and estimate the price elasticity of groundwater demand to be $-1.12$. These elasticities are much larger than previous estimates in the electricity and groundwater literatures.

We then explore the mechanisms underlying farmers' groundwater demand response. We develop a stylized model of farmer decisions under changing water prices, which highlights the potential relevance of crop switching or fallowing. We then empirically evaluate the role of crop switching in a series of simple empirical tests. We find evidence consistent with crop switching being the primary mechanism driving our estimated demand elasticities. First, our monthly and annual elasticity estimates are similar, suggesting that within-year surface water substitution is not driving our results. Second, farmers respond strongly on the extensive margin of groundwater pumping, which is challenging from an agronomic perspective without switching crops. Third, we find the largest elasticities, $-2.65$ for electricity and $-2.63$ for groundwater, amongst farmers who switch between annual and perennial crops during our sample. Finally, we estimate that a 1 percent increase in groundwater or electricity prices leads to a 4 percent decrease in the likelihood of planting annual crops (as opposed to perennial crops or fallowing), though this estimate is noisy.

Our results have important implications for groundwater management policies, such as California's Sustainable Groundwater Management Act (SGMA). As global aquifers continue to fall to new historic lows (\textcite{famiglietti2014}), governments seek policies to manage common-pool groundwater resources and disincentivize farmers from ``racing to the bottom.'' While the type of policy may vary across settings (e.g.\ \textcite{ryan2020}), it is important to understand both the extent to which farmers are likely to respond by reducing groundwater extraction and any (unintended) consequences of their demand response. Our large elasticity estimates imply that incremental groundwater management efforts have the potential to yield meaningful improvements in groundwater sustainability. The crop switching mechanism also signals a potential second-best benefit of groundwater management: if unpriced groundwater has led to suboptimal allocation of land to crops, where farmers fail to internalize the externalities of their own groundwater extraction, then pricing groundwater may reduce crop misallocation by inducing farmers to plant fewer acres of low-value, water-intensive crops.

In ongoing work, we are extending these results along three dimensions. First, we are incorporating data from Southern California Edison, thereby covering nearly all of California's farmland. Second, we are studying how changes in surface water availability influence both groundwater extraction and cropping decisions, by incorporating data from \textcite{hagerty2019}. Third, we aim to quantify the effects of droughts on electricity demand, groundwater extraction, and agricultural land use.


%Using these elasticity estimates, we quantify the ``pumping cost'' externality---or the extent to which farm $i$ increases its neighbors' groundwater extraction costs by removing a marginal unit of water from their shared aquifer.
%Our preliminary results suggest that the magnitude of this externality is likely small relative to the private costs of groundwater pumping, and that a small corrective tax could yield substantial welfare gains for California farmers.

%These results are still preliminary, and future versions of this paper will incorporate more sophisticated hydrogeological assumptions to more accurately characterize the spatial nature of the pumping cost externality. We also plan to use detailed geospatial data in order to aggregate up from electricity service points to farms, as well as high-resolution satellite imagery to classify farms by crop type. This will allow us to estimate heterogeneous elasticities by crop type, and to incorporate these heterogeneous elasticities into our externality calculations. A richer version of Table \ref{tab:externality_calcs} will differentiate between annuals vs.\ perennials, crops with high vs.\ low water intensity, and high- vs.\ low-value crops---letting us characterize the extent to which the pumping cost externality contributes to misallocation in crop choice. Future work will use quasi-experimental variation in farmers' pumping capital to estimate the long-run elasticity of demand for groundwater.