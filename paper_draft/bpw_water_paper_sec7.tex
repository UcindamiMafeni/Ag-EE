%
% Conclusion
%
In this paper, we estimate how a key sector---agriculture---is likely to respond to environmental regulation which will increase the costs of an essential input---groundwater---in the setting of California, one of the most productive farming regions in the world. While California accounts for 18 percent of total U.S.\ crop value, its farmers are heavily dependent on groundwater-based irrigation. As a result of overuse, water levels in California's underground aquifers have fallen substantially, with 21 of the state's groundwater basins now deemed ``critically overdrafted''.

In an effort to prevent aquifer collapse, California policymakers have recently passed the state's first comprehensive groundwater legislation, the Sustainable Groundwater Management Act (SGMA), which will require large reductions in groundwater use. Regardless of the exact implementation approach, these regulations will raise the costs of groundwater for California's farmers.

In order to understand how farmers will respond to increases in water costs, we must overcome measurement and identification challenges: groundwater is typically neither priced nor measured, and groundwater costs are typically not randomly assigned. We leverage the insight that electricity is the key marginal input into the groundwater production function. We use a novel restricted-access dataset on farmers' electricity consumption and groundwater pump efficiencies, combined with government measurements of groundwater depths and satellite-derived land use designations for the universe of farmers in the Pacific Gas and Electric (PGE) utility service territory, which covers the majority of the farmland in the state. We leverage exogenous variation in electricity tariffs over time to estimate farmers' price elasticity of demand for electricity, and find a surprising large elasticity estimate of $-1.17$. We then use the physics of groundwater pumping to compute groundwater costs and groundwater quantities for each pump in our sample, and we estimate the price elasticity of groundwater demand to be $-1.12$. These elasticities are much larger than previous estimates in the electricity and groundwater literatures.

We then explore the mechanisms underlying farmers' groundwater demand response. We find evidence consistent with crop switching as the primary mechanism driving our estimated demand elasticities. First, farmers do not appear to respond to large within-day switches in price, meaning that changing water use on existing crops is unlikely to explain our results. Second, we find similar elasticities when we restrict our sample to months around pump tests, making it unlikely that pumping capital upgrades are driving our estimates. Third, because surface water tends to be substantially cheaper than groundwater, and because we estimate similar elasticities at the monthly and annual level, within-year surface water substitution is not likely to be the primary mechanism. Finally, we find a large semi-elasticity on the extensive margin of groundwater use, showing that an key means of adjustment is to cease pumping altogether---consistent with crop switching or fallowing.

We build on this evidence by using a discrete choice model to estimate the impact of increasing groundwater costs on crop choice. We find that higher groundwater costs cause farmers to increase acreage in fruit and nut perennials and increase fallowing, and to decrease acreage in annuals and other perennials. We simulate a counterfactual groundwater tax to estimate the impacts of potential groundwater pricing policies on land use in California. We find that a moderate \$10 per acre-foot tax---approximately the price increase our elasticity estimates imply would be required to meet California's lower-bound target of 20 percent curtailment---would lead farmers to reallocate nearly 8 percent of land to a different crop type. If the planned curtailments under SGMA reflect the true externality from groundwater extraction, extrapolating our estimates from our sample to the rest of the state suggests that nearly 2 million acres of California cropland may be misallocated due to unpriced groundwater.


%Our results have important implications for groundwater management policies, such as California's Sustainable Groundwater Management Act (SGMA). As global aquifers continue to fall to new historic lows (\textcite{famiglietti2014}), governments seek policies to manage common-pool groundwater resources and disincentivize farmers from ``racing to the bottom.'' While the type of policy may vary across settings (e.g.\ \textcite{ryan2020}), it is important to understand both the extent to which farmers are likely to respond by reducing groundwater extraction and any (unintended) consequences of their demand response. Our large elasticity estimates imply that incremental groundwater management efforts have the potential to yield meaningful improvements in groundwater sustainability. The crop switching mechanism also signals a potential second-best benefit of groundwater management: if unpriced groundwater has led to suboptimal allocation of land to crops, where farmers fail to internalize the externalities of their own groundwater extraction, then pricing groundwater may reduce crop misallocation by inducing farmers to plant fewer acres of low-value, water-intensive crops. {\color{blue} We find that this impact is potentially quite large:} {\color{magenta}across all of California agriculture, nearly two million acres may be currently misallocated.}


