%
% Abstract

\noindent
Groundwater is a key resource for agricultural production globally. Both increasingly rapid drawdowns of aquifers as well the policies intended to increase aquifer sustainability increase costs to agricultural producers, with unknown consequences. In this paper, we provide the first large-scale empirical estimates of how farmers respond to changes in groundwater costs in one of the world's most valuable agricultural areas: California. To do this, we assemble a novel dataset that combines (i) detailed restricted-access microdata on farmers' electricity consumption, (ii) rich data from technical audits of these farmers' pump efficiencies, (iii) measurements of groundwater depths in California aquifers, and (iv) satellite-derived measures of crop cover. For identification, we leverage exogenous variation in the price of electricity, a key marginal input into the groundwater production function. We find that farmers are very price responsive: we estimate large price elasticities of demand for electricity ($-1.17$) and groundwater ($-1.12$). We demonstrate that crop switching and fallowing are the main channel through which farmers respond to increases in groundwater costs. Using a discrete choice model, we estimate that a counterfactual \$10 per-acre-foot groundwater tax---approximately the price increase required to meet California's sustainability targets---would lead farmers to reallocate 8 percent of cropland, with increases in fallowing and high-value fruit and nut perennials, and decreases in annual crops and low-value perennials. 
 %We find evidence that crop switching and fallowing is the primary mechanism behind these large elasticity estimates. {\color{magenta}Our results imply that a moderate groundwater tax of \$10 per acre-foot would cause farmers to reallocate nearly 8\% of cropland to a different crop type.} These results suggest that groundwater management policy may have important impacts on the markets for agricultural products.
%
%We then {\color{magenta}investigate possible mechanisms and find empirical evidence that the primary mechanism for these large elasticities is crop switching and fallowing in response to increased prices for electricity and groundwater.}
%{\color{blue}test two main mechanisms behind this large elasticity: substitution between groundwater and surface water, and crop switching. We find [xxx].  }\\
\\

