%
% RESULTS 
%

\subsection{Electricity demand}

Table \ref{tab:elec_regs_main} reports  results from estimating Equation \ref{eq:reg_elec}. Column (1)  does not instrument for marginal electricity price, yielding an unidentified $\hat\beta$ estimate. 
Column (2) instruments using unit $i$'s within-category default marginal price, which eliminates bias from farmers choosing their own electricity tariffs. The direction of this bias is not obvious \emph{ex ante}, because farmers are choosing between tariffs with both volumetric (\$/kWh) and fixed (\$/kW) price components.\footnote{
PGE tariffs with relatively high volumetric (i.e.\ marginal) prices tend to have relatively low fixed prices, and vice versa. Two farmers with the same average electricity consumption may optimally choose different tariffs. Suppose farmer A operates a 300 hp pump for 50 hours per month, while farmer B operates 50 hp pump for 300 hours per month. Farmer A should prefer a low fixed price and a high volumetric price, while farmer B should prefer a high fixed price and a low marginal price.
}
Comparing Columns (1) vs.\ (2), we see that on average, farmers with higher electricity consumption tend to choose  tariffs with relatively low fixed charges per kW and relatively high prices per kWh.

Column (3) eliminates the other source of price endogeneity by interacting unit fixed effects with indicators for (i) small pumps ($<35$ hp), (ii) large pumps ($\ge35$ hp), and (iii) auxiliary internal combustion engines. While only 5 percent of units shift across tariff categories due to changes in their physical capital, the resulting simultaneous changes in $Q^{\text{elec}}_{it}$ and $P^{\text{elec}}_{it}$ induce substantial bias in Column (2) point estimate. Column (3) reports our preferred estimate of $-1.17$, after having purged both sources of price endogeneity.\footnote{If we modify Column (3) by interacting month-of-sample fixed effects with bins, we recover a point estimate of $-1.15$ with a standard error of $0.16$. This shows that differential trends in pump size are not biasing our results.}

The magnitude of this elasticity estimate is surprisingly large, considering that electricity demand tends to be extremely inelastic in other contexts.
The literature on electricity demand has focused heavily on the residential sector, and recent estimates have found elasticities of $-0.08$ to $-0.48$ in the short run (\textcite{reiss2005}; \textcite{alberini2011}; \textcite{fell2014}) and $-0.09$ to $-0.73$ in the medium-to-long run (\textcite{alberini2011}; \textcite{ito2014}; \textcite{deryugina2018}).\footnote{
These estimates uses monthly or annual variation in electricity prices, which aligns with our empirical strategy. Other studies leverage hourly variation in electricity prices have estimated electricity demand elasticities ranging from $-0.03$ to $-0.25$ (\textcite{wolak2011}; \textcite{jessoe2014}; \textcite{fowlie2018}; \textcite{ito2018}).} 
Fewer  estimates exist for commercial or industrial electricity demand. \textcite{paul2009} estimate commercial/industrial elasticities of $-0.11$ to $-0.16$ in the short run, and $-0.29$ to $-0.40$ in the long run. \textcite{jessoe2015} find no demand response to dynamic pricing in these sectors, while \textcite{blonz2016} estimates elasticities of $-0.08$ to $-0.22$ using hourly price variation for PGE's small commercial/industrial customers.
To our knowledge, we provide the first large-scale estimates of electricity demand elasticities in the agricultural sector.


Columns (4)--(6) report three additional elasticity estimates, each intended to assuage any remaining concerns over price endogeneity. Column (4)  includes separate year fixed effects for each water basin and each water district, to control for potential time-varying confounders related to water depth or surface water availability. The resulting point estimate of $-1.02$ is similar, albeit slightly attenuated. Column (5) instruments with the 6- and 12-month lags of the default price (rather than the contemporaneous default price), to account for potential endogeneity in the timing of PGE's smart meter rollout.\footnote{
Recall that farmers may shift across tariff categories (inducing changes to their within-category default price) due to \emph{either} changes in their physical capital \emph{or} the installation of a smart meter.
} This yields a nearly identical point estimate, implying that farmers' electricity consumption did not meaningfully change in anticipation of a smart meter installation. Finally, Column (6) adds 11,175 unit-specific linear time trends, to confirm that we are not identifying $\hat \beta$ solely off of monotonic trends in price and quantity. The resulting point estimate of $-0.76$ is attenuated, as linear trends remove much of the (good) variation in electricity prices over time. Even so, we still find a tightly estimated elasticity that is substantially larger than virtually all previous estimates for the elasticity of electricity demand.


\subsection{Groundwater demand}

Table \ref{tab:water_regs_main} presents our results for estimating farmers' groundwater demand. Each column estimates Equation (\ref{eq:reg_water}) using our preferred strategy for identifying the elasticity with respect to the electricity price: instrumenting for $\log\big(P_{it}^{\text{elec}}\big)$ with within-category default prices, and interacting unit fixed effects with indicators for each category of physical pumping capital. Note that we report $\hat\beta^{\text{e}}$ and $\hat\beta^{\text{w}}$, where the latter subtracts 1 from the regression coefficient on $\log\big(\widehat{\tfrac{{\text{kWh}}}{\text{AF}}}_{it}\big)$. We interpret each coefficient as the elasticity of demand for groundwater with respect to one component of the price of groundwater, holding the other component constant.

Column (2) reports our preferred estimates of $\hat\beta^{\text{e}}$ and $\hat\beta^{\text{w}}$, where we instrument for $\log\big(\widehat{\tfrac{{\text{kWh}}}{\text{AF}}}_{it}\big)$ with logged groundwater depth in month $t$ averaged across unit $i$'s groundwater basin. Comparing $\hat\beta^{\text{w}}$ in Columns (2) vs.\ (1), instrumenting with average depth appears to alleviate bias due to measurement error in $\log\big(\widehat{\tfrac{{\text{kWh}}}{\text{AF}}}_{it}\big)$.\footnote{
We discuss three potential sources of bias in $\beta^{\text{w}}$ in Section \ref{sec:empirics_water}: (i) endogenous changes to pumping technologies, (ii) the mechanical relationship between extraction and depth at a given location, and (iii) measurement error. Bias from (i) and (ii) appear unlikely, as they should bias our $\beta^{\text{w}}$ away from zero, rather than towards zero.
}
The exclusion restriction requires that unit $i$'s pumping behavior have no contemporaneous impact on basin-wide average groundwater depths. Such feedback effects between the dependent variable and the instrument would be extremely unlikely for three reasons: (i) unit $i$ is small relative to the geographic footprint of its groundwater basin; (ii) thousands of other pumpers are also extracting from the same basin; (iii) basin-wide average groundwater levels do not instantaneously reequilibrate after extraction at one point in space. Column (3) restricts the sample to the 3 largest groundwater basins, each of which has over 1,000 units in our estimation sample.\footnote{These basins are the San Joaquin Valley, the Sacramento Valley, and the Salinas Valley. The number of agricultural groundwater pumpers in each basin is likely much larger, as our estimation sample comprises only the subset of PGE customers that we can confident match to an APEP-subsidized pump test.
}
The resulting $\hat\beta^{\text{w}}$ estimate is quite similar, which should assuage concerns that the instrument is invalid due to a few large farms located in very small groundwater basins.

The magnitudes of our $\hat\beta^{\text{e}}$ estimates are quite similar to results from the electricity-only regressions, especially comparing $\hat\beta^{\text{e}} = -1.21$ from Column (1) of Table \ref{tab:water_regs_main} the analogous estimate of $\hat\beta = -1.17$ from Column (3) of Table \ref{tab:elec_regs_main}. This is not surprising, since Equation (\ref{eq:reg_water}) simply adds one regressor to Equation (\ref{eq:reg_elec}). However, $\hat\beta^{\text{e}}$ is surprisingly close to our instrumented $\hat\beta^{\text{w}}$ estimate ($-1.39$ vs.\ $-1.37$). This implies that a 1 percent change in the effective price of groundwater has the \emph{same} effect on farmers' pumping behavior, whether that change comes via their marginal electricity price or via their pump's kWh/AF conversion factor. It also suggests that farmers are quite attentive to their true costs of pumping, and that they reoptimize their pumping behavior identically in response to either type of price variation---as Neoclassical theory would predict.

Similar to our elasticity estimates for electricity, our groundwater elasticity estimates are also quite large relative to the existing literature. Recent studies have also exploited variation in energy prices, but yielding far smaller magnitudes: \textcite{hendricks2012} find an elasticity of $-0.10$, and \textcite{pfeiffer2014} find an elasticity of $-0.27$ (both for agricultural groundwater in Kansas). \textcite{bruno2018} estimate demand elasticities of $-0.17$ to $-0.22$ within the Coachella Valley of California, which is a unique setting where groundwater extraction is directly priced. Previous studies have also estimated farmers' elasticity of demand for surface water, most notably 
\textcite{hagerty2018}, who finds an elasticity of $-0.23$ for surface water in California agriculture. While estimates of surface water demand are often as large as $-0.8$ in specific locations (\textcite{schoengold2006}; \textcite{hagerty2018}), we find agricultural groundwater demand to be even more elastic \emph{on average}.\footnote{
Estimates for urban water demand have found similar elasticities, ranging from $-0.10$ to $-0.76$ (\textcite{nataraj2011}; \textcite{ito2013}; \textcite{baerenklau2014}; \textcite{wichman2014}; \textcite{buck2016}; \textcite{wichman2016}; \textcite{hagerty2018}).
}
Substitution between groundwater and surface water is likely a major factor explaining the large magnitudes of our elasticities estimates.

Columns (4)--(6) report three alternate versions of our preferred estimates in Column (2). First, to account for the inherent tradeoff between spatial density vs.\ temporal frequency of groundwater measurements, Column (4)  re-estimates  Equation (\ref{eq:reg_water}) using groundwater data rasterized at the quarterly (rather than monthly) level. Whereas our preferred monthly rasters are able to capture groundwater measurements at greater temporal frequency, quarterly rasters have greater accuracy in the cross-section by incorporating more distinct measurement sites. The resulting $\hat\beta^{\text{w}}$ estimate increases in magnitude, however the average depth instrument has less predictive power at the (coarser) quarterly level.
Column (5) includes water basin by year and water district by year fixed effects, yielding only slightly attenuated point estimates despite eliminating much of the variation in the average depth instrument.
In Column (6), we instrument with 6- and 12-month lags of average depth (rather than contemporaneous depth), as it is possible (albeit unlikely) that farmers pump less in months with lower groundwater levels for some reason other than pumping costs. These lagged instruments substantially increase $\hat\beta^{\text{e}}$ and $\hat\beta^{\text{w}}$; however, the small first stage $F$-statistic indicates a weak instrument, and we interpret these results with caution.

Table \ref{tab:water_regs_combined} reports results from estimating Equation (\ref{eq:reg_water_combined}), with groundwater quantity as the dependent variable, and instrumenting for the composite groundwater price with default electricity prices. While these estimates identify $\hat\beta$ using \emph{only} variation in default electricity prices, Table \ref{tab:water_regs_main} demonstrates that farmers respond almost identically to variation in \emph{either} component of their effective groundwater price.  The resulting point estimates are quite similar to, but slightly smaller than, our electricity demand estimates.\footnote{
Again, it is not surprising that Equations (\ref{eq:reg_elec}) and (\ref{eq:reg_water_combined}) yield similar point estimates, since $Q^{\text{water}}$ and $P^{\text{water}}$ are multiplicative transformations of $Q^{\text{elec}}$ and $P^{\text{elec}}$ and both specificaitons are log-log.
} 
Interestingly, combining $P^{\text{water}}_{it}$ into a single regressor removes much of the variation used to estimate separate coefficients in Table \ref{tab:water_regs_main}. This is because electricity prices and groundwater depths are seasonally correlated---groundwater levels are lowest (making pumping more expensive) in the winter months, when electricity prices are also low. This likely explains why $\hat\beta$ estimates in Table \ref{tab:water_regs_combined} are smaller than $\hat\beta^{\text{e}}$ and $\hat\beta^{\text{w}}$ estimates in Table \ref{tab:water_regs_main}. 



