%
% RESULTS 
%

\subsection{Electricity demand}
\label{sec:results_elec_demand}

Table \ref{tab:elec_regs_main} reports results from estimating Equation (\ref{eq:reg_elec}). Column (1) presents the OLS results without instrumenting for marginal electricity price, resulting in a potentially biased $\hat\beta$ estimate. 
Column (2) instruments using unit $i$'s within-category default marginal price, which eliminates bias from farmers choosing their own electricity tariffs. The direction of this bias is not obvious \emph{ex ante}, because farmers are choosing between tariffs with both volumetric (\$/kWh) and fixed (\$/kW) price components.\footnote{
PGE tariffs with relatively high volumetric (i.e.\ marginal) prices tend to have relatively low fixed prices, and vice versa. Two farmers with the same average electricity consumption may optimally choose different tariffs. Suppose farmer A operates a 300 hp pump for 50 hours per month, while farmer B operates 50 hp pump for 300 hours per month. Farmer A should prefer a low fixed price and a high volumetric price, while farmer B should prefer a high fixed price and a low marginal price.
}
Comparing Columns (1) vs.\ (2), we see that on average, farmers with higher electricity consumption tend to choose  tariffs with relatively low fixed charges per kW and relatively high prices per kWh.

Column (3) eliminates the other potential source of price endogeneity---farmers switching across rate categories---by interacting unit fixed effects with indicators for (i) small pumps ($<35$ hp), (ii) large pumps ($\ge35$ hp), and (iii) auxiliary internal combustion engines. While only 5 percent of units shift across tariff categories due to changes in their physical capital, the resulting simultaneous changes in $Q^{\text{elec}}_{it}$ and $P^{\text{elec}}_{it}$ induce substantial bias in Column (2) point estimate. Column (3) reports our preferred estimate of $-1.17$, after having purged both sources of price endogeneity.\footnote{Appendix Table~\ref{tab:elec_regs_month_bin_hp_fes} presents results from specifications where we interact month-of-sample fixed effects with bins of horsepower, kW, or operating pump efficiency. We recover extremely similar point estimates, ($-1.15$ with a standard error of $0.16$ for horsepower), indicating that differential trends in pump size are not biasing our results.}

Columns (4)--(6) report three additional elasticity estimates, each intended to assuage any remaining concerns over price endogeneity. Column (4) includes separate year fixed effects for each water basin and each water district, to control for potential time-varying confounders related to groundwater depth or surface water availability. The resulting point estimate of $-0.95$ is similar, albeit slightly attenuated. Column (5) instruments with the 6- and 12-month lags of the default price (rather than the contemporaneous default price), to account for potential endogeneity in the timing of PGE's smart meter rollout.\footnote{
Recall that farmers may shift across tariff categories (inducing changes to their within-category default price) due to either changes in their physical capital or the installation of a smart meter.
} This yields a nearly identical point estimate, implying that farmers' electricity consumption did not meaningfully change in anticipation of a smart meter installation. Finally, Column (6) adds 11,173 unit-specific linear time trends, to confirm that we are not identifying $\hat \beta$ solely off of monotonic trends in price and quantity. The resulting point estimate of $-0.76$ is attenuated, as linear trends remove much of the (exogenous) variation in electricity prices over time, but still indicates that farmers respond strongly to changes in electricity prices. 


In Appendix~\ref{app:tab_fig}, we present a series of sensitivity checks on this main result: instrumenting with modal within-category tariffs, rather than default tariffs (Appendix Table~\ref{tab:elec_regs_modal_tariff}); interacting month-of-sample fixed effects with bins of pump horsepower, measured load, and efficiency (Appendix Table~\ref{tab:elec_regs_month_bin_hp_fes}); interacting month-of-sample fixed effects with geographic fixed effects (Appendix Table~\ref{tab:elec_regs_month_int_fes}); and adding weather controls (Appendix Table~\ref{tab:elec_water_regs_weather}). Across all specifications, we find large, statistically significant elasticities that are quantitatively similar to our preferred specification in Column (3) of Table~\ref{tab:elec_regs_main}. Appendix Table~\ref{tab:elec_regs_ihs_logs} reveals that while our estimates are not sensitive to our choice of functional form, they are sensitive to the inclusion of observations with zero electricity consumption. If we exclude unit-months with zero electricity consumption, our elasticity estimate attenuates to $-0.31$. This is consistent with farmers responding to high electricity prices by changing their cropping patterns---a mechanism that we explore in greater detail in Section~\ref{sec:mechanisms} below. 



Across specifications, we find large, precisely-estimated elasticities. The magnitude of these estimates is surprisingly large, considering that electricity demand tends to be quite inelastic in other contexts.
The literature on electricity demand has focused heavily on the residential sector, and recent estimates have found elasticities of $-0.08$ to $-0.48$ in the short run (\textcite{reiss2005}; \textcite{alberini2011}; \textcite{fell2014}) and $-0.09$ to $-0.73$ in the medium-to-long run (\textcite{alberini2011}; \textcite{ito2014}; \textcite{deryugina2018}).\footnote{
These estimates use monthly or annual variation in electricity prices, which aligns with our empirical strategy. Studies that leverage hourly variation in electricity prices have estimated electricity demand elasticities ranging from $-0.03$ to $-0.25$ (\textcite{wolak2011}; \textcite{jessoe2014}; \textcite{fowlie2018}; \textcite{ito2018}).} 
Fewer estimates exist for commercial or industrial electricity demand. \textcite{paul2009} estimate commercial/industrial elasticities of $-0.11$ to $-0.16$ in the short run, and $-0.29$ to $-0.40$ in the long run. \textcite{jessoe2015} find no demand response to dynamic pricing in these sectors, while \textcite{blonz2016} estimates elasticities of $-0.08$ to $-0.22$ using hourly price variation for PGE's small commercial/industrial customers.
To our knowledge, we provide the first large-scale estimates of electricity demand elasticities in the agricultural sector.


\subsection{Groundwater demand}
Table \ref{tab:water_regs_combined} presents our estimates of the price elasticity of farmer's groundwater demand. In each column, we estimate Equation (\ref{eq:reg_water_combined}), including unit-by-himonth-of-year fixed effects, month-of-sample fixed effects, and interactions between unit fixed effects and physical capital. In Column (1), we present OLS results. As with the electricity results in Table~\ref{tab:elec_regs_main}, we find a smaller elasticity with the OLS ($-0.88$) than with our instrumental variables approach in the following columns. 
In Column (2), our preferred specification, we instrument for the price of groundwater using  using unit $i$'s within-category default marginal price, eliminating bias from farmers choosing their electricity tariff. We estimate a price elasticity of groundwater demand of $-1.12$. 

Columns (3)--(6) present a series of sensitivity checks around this central estimate. Column (3) restricts the sample to the three largest groundwater basins, each of which has over 1,000 units in our estimation sample.\footnote{These basins are the San Joaquin Valley, the Sacramento Valley, and the Salinas Valley. The number of agricultural groundwater pumpers in each basin is likely much larger, as our estimation sample comprises only the subset of PGE customers that we can match to an APEP-subsidized pump test.} The resulting $\hat\beta$ estimate is quite similar, which should assuage concerns that the instrument is invalid due to a few large farms located in very small groundwater basins. 
In Column (4), we convert from electricity to groundwater by re-calculating $\widehat{\tfrac{{\text{kWh}}}{\text{AF}}}_{it}$ using groundwater data rasterized at the quarterly (rather than monthly) level; this addresses the inherent tradeoff between spatial density vs.\ temporal frequency of groundwater measurements, and has little effect on our results. 
Column (5) includes water-basin-by-year and water-district-by-year fixed effects, yielding a slightly attenuated point estimate ($-0.90$). In Column (6), we instrument with 6- and 12-month lags of the default within-category electricity price (rather than contemporaneous prices); we again find a quantitatively similar estimate of $-1.14$.

The point estimates in Table~\ref{tab:water_regs_combined} are quite similar to our electricity demand estimates.\footnote{
It is not surprising that Equations (\ref{eq:reg_elec}) and (\ref{eq:reg_water_combined}) yield similar point estimates, since $Q^{\text{water}}$ and $P^{\text{water}}$ are multiplicative transformations of $Q^{\text{elec}}$ and $P^{\text{elec}}$, and both specifications use the same two-stage least squares model.
}   Appendix~\ref{app:gw_demand} presents the results from an alternative approach where we separately estimate the price elasticity of demand for groundwater with respect to electricity prices vs.\ kWh-to-AF conversion factors. In our preferred specification, we estimate larger elasticities than in Table~\ref{tab:water_regs_combined}: $-1.27$ for groundwater price changes induced by electricity price changes, and $-1.51$ for groundwater price changes induced by kWh-to-AF changes. These estimates are not statistically different from each other, which signals that farmers respond similarly to both types of pumping cost changes---as Neoclassical theory would predict, if farmers are indeed rationally optimizing over groundwater as an agricultural input.

Appendix~\ref{app:tab_fig} presents additional robustness checks: sensitivities to how we construct the $\widehat{\tfrac{{\text{kWh}}}{\text{AF}}}_{it}$ conversion factor (Appendix Table~\ref{tab:water_kwhaf_sensitivities}); interacting month-of-sample fixed effects with geographic fixed effects (Appendix Table~\ref{tab:water_regs_month_int_fes}); including weather controls (Appendix Table~\ref{tab:elec_water_regs_weather}); and sensitivities by assignment to CLUs (Appendix Table~\ref{tab:water_clu_sensitivities}). We find large, statistically significant elasticities that are quantitatively similar to our preferred point estimate from Column (2) of Table~\ref{tab:water_regs_combined} across specifications. As with the electricity regressions described in Section~\ref{sec:results_elec_demand}, we find that our groundwater results are sensitive to the inclusion of observations with zero groundwater pumping: excluding unit-months with zero groundwater extraction attenuates our elasticity to $-0.30$ (Appendix Table~\ref{tab:water_regs_ihs_logs}). This suggests that an important method of adjustment to high groundwater prices is halting pumping, which is again consistent with fallowing or crop switching as a mechanism. We examine this possibility in Section~\ref{sec:mechanisms} below.\footnote{Below, Table~\ref{tab:elec_water_intens_extens} further corroborates this intensive-vs.-extensive margin result at the annual level.}

As with our elasticity estimates for electricity, our groundwater elasticity estimates are quite large relative to the existing literature. Recent studies have also exploited variation in energy prices but have yielded far smaller magnitudes: \textcite{hendricks2012} find an elasticity of $-0.10$, and \textcite{pfeiffer2014} find an elasticity of $-0.27$ (both for agricultural groundwater in Kansas). \textcite{bruno2018} estimate demand elasticities of $-0.17$ to $-0.22$ within the Coachella Valley of California, which is a unique setting where groundwater extraction is directly priced.\footnote{
Prior work on urban water demand, a setting in which researchers also observe prices and quantities for water, has found similar elasticities, ranging from $-0.10$ to $-0.76$ (\textcite{nataraj2011}; \textcite{ito2013}; \textcite{baerenklau2014}; \textcite{wichman2014}; \textcite{buck2016}; \textcite{wichman2016}; \textcite{hagerty2018}). Previous studies have also estimated farmers' elasticity of demand for surface water, most notably 
\textcite{hagerty2018}, who finds an elasticity of $-0.23$ for surface water in California agriculture. While estimates of surface water demand are often as large as $-0.80$ in specific locations (\textcite{schoengold2006}; \textcite{hagerty2018}), we find agricultural groundwater demand to be even more elastic on average.
}




