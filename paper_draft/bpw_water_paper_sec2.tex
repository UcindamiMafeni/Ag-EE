%
% BACKGORUND 
%

\subsection{Agriculture in California}
California is a key player in global agricultural production, and represented 17 percent of total U.S. crop value in 2016 (\textcite{crs2015}). California's 77,000 farms produce over 400 commodities, including close to half of all fruits, nuts, and vegetables grown in the United States. In fact, California is the sole domestic producer of many high-value crops, including almonds, artichokes, olives, and walnuts (\textcite{cdfa2011}).  

%\subsubsection{Agricultural water use}
Water is an essential input for California's agricultural production. Nearly 80 percent of the state's annual water consumption occurs in the agricultural sector, where crop irrigation is the primary end use. California has nearly 8.3 million harvested acres of cropland, 7.9 million of which are irrigated (\textcite{crs2015}). Farmers may have access to surface water and/or groundwater, and each water source is governed by a complex system of rights.\footnote{See \textcite{sawyers2007} for more details.} 


\begin{comment}
\paragraph{Surface water}
{\color{blue}
Approximately 40 percent of California's surface water is used in agriculture. 61 percent of irrigation water comes from surface sources, with groundwater making up the remaining 39 percent {\color{blue}(CA DWR 2015, cite in hagerty p 7)}. Most farms with access to surface water obtain it via irrigation districts.\footnote{Irrigation districts were established between 1860 and 1950, and their boundaries have remained essentially fixed over time. Though some individual farms do have their own water entitlements, the vast majority of these allocations belong to districts. {\color{blue} would be good to get a stat on this}.} These agricultural cooperatives divert water from large rivers and canals and distribute this water to farmers, with individual farmers receiving water proportional to their acreage within the district (\textcite{schlenker et al 2007}). 

Irrigation districts obtain water through two main sources: project contracts and water rights. Project contracts allow irrigation districts to withdraw water from state- and federally-operated canals.\footnote{There are three such water projects in California: the State Water Project, the Central Valley Project, and the Lower Colorado Project. The State Water Project is managed by the California government; the other two are federally managed.} Water rights allow irrigation districts to withdraw water from nearby rivers and streams. In California, These rights are under a prior appropriation system, meaning that rights are established by the first claimant, and held as long as water is continuously used.  

Both water rights and project contracts provide irrigation districts with fixed \emph{maximum} appropriations over time, but annual allocations can vary across years. Under drought conditions, water rights are allocated according to the age of the right. Senior rightsholders, with rights established prior to 1914, are entitled to their full allocation in all years. Junior rightsholders, with rights established in and after 1914, are ranked in order of age of the right, and allocated water until the maximum allotment has been distributed. {\color{blue} WE SHOULD TRIPLE CHECK TO MAKE SURE THIS IS ACTUALLY ACCURATE.} Project contracts also vary their allocations year by year, using an algorithm which depends only on environmental conditions. {\color{blue} something about how this works by giving you a share of your max allotment.} Information on allocations is revealed to farmers well in advance of the growing season.

For a more detailed description of surface water rights in California, see \textcite{hagerty2019}.}
\end{comment}

\paragraph{Surface water}
Approximately 40 percent of California's surface water is used in agriculture. 61 percent of irrigation water comes from surface sources, with groundwater making up the remaining 39 percent (\textcite{cdwr2015}). Surface water rights in California follow strict rules. Most farms with access to surface water obtain it via irrigation districts.\footnote{Irrigation districts were established between 1860 and 1950, and their boundaries have remained essentially fixed over time. Though some individual farms do have their own water entitlements, the vast majority of these allocations belong to districts. These agricultural cooperatives divert water from large rivers and canals and distribute this water to farmers, with individual farmers receiving water proportional to their acreage within the district (\textcite{schlenker2007}). For a more detailed description of surface water rights in California, see \textcite{hagerty2019}.} In addition to obtaining surface water from individual rights or irrigation districts, farmers have a limited ability to purchase water on the open market. However, these trades constitute only a very small share of total water deliveries, and the prices are extremely high (\textcite{hagerty2018}).


\paragraph{Groundwater}
In normal weather conditions, groundwater supplies 30 to 40 percent of all water end uses in California. However, this rises to close to 60 percent in drought years, when surface water is unusually scarce (\textcite{cdwr2014}).
By contrast to the strictly defined surface water rights, agricultural groundwater rights tend to be far more vague. The typical groundwater right in California is ``overlying,'' meaning that landowners whose property sits above an aquifer have the right to extract the underlying groundwater.\footnote{
There are also ``appropriative'' groundwater rights, for users who do not own land above the aquifer, but these rights are lower-priority than the overlying rights. Appropriative rights holders may only exercise these rights in the case of a surplus.}
The vast majority of groundwater use is unmetered, and users face no variable costs of extraction beyond the energy costs of pumping (\textcite{bruno2018}).\footnote{There are limited exceptions to this rule: a few irrigation districts impose a per-unit price on groundwater, but this is rare (\textcite{bruno2018}).} Hence, a farmer may extract as much groundwater as he chooses, conditional on owning the overlying property rights. 

Many of California's groundwater basins are ``overdrafted,'' meaning that, annually, withdrawals exceed replenishment. As of 2017, some agricultural regions were facing overdraft of 2 million acre-feet annually. This has led to a substantial decline in groundwater levels in the Central Valley, most notably in the Tulare and San Joaquin groundwater basins, which, combined, lost more than 135 million acre-feet of groundwater since 1925.\footnote{See: \url{https://www.ppic.org/publication/groundwater-in-california/}} In 2014, the state faced a severe drought, with groundwater levels reaching historic lows in many portions of the state. 21 of the state's 515 groundwater basins are considered ``critically overdrafted.''

In response to drought conditions, in September 2014, California lawmakers passed sweeping groundwater legislation. The Sustainable Groundwater Management Act (SGMA), consists of three bills, and represents  represents the first statewide regulatory scheme to mitigate over-extraction of groundwater. AB 1739 enables the California Department of Water Resources (DWR) or local groundwater sustainability agencies (GSAs) the ability to charge fees for groundwater extraction, and requires GSAs to prepare groundwater sustainability plans (GSPs). SB 1319 authorizes GSAs to implement these GSPs. SB 1168 requires that uses of groundwater be both reasonable and beneficial, and enables GSAs and the DWR to require groundwater monitoring. In addition, Proposition 1 provided \$100 million in funding to support sustainable groundwater management. SGMA represents the future of groundwater management in California. However, the GSPs will not be finalized until 2020 at the earliest, leaving farmers free to extract at will.

\subsection{Electricity for pumping}
Electricity is an essential input to groundwater pumping. The California Energy Commission reports that water use accounts for 19 percent of California's electricity consumption, and close to 8 percent of the state's energy is used on farms (\textcite{cec2005}). The state's investor-owned utilities spend nearly \$50 million annually to improve energy efficiency in the agricultural sector. This makes water use, and agricultural water use in particular, a key component of California's energy policy goals.

To estimate groundwater demand, we exploit the fact that electricity is a major determinant of pumping costs. Several previous papers have used variation in energy costs to estimate the price elasticity of groundwater demand (\textcite{hendricks2012}, \textcite{pfeiffer2014}, \textcite{badiani2015}, and \textcite{mieno2017}). However, \textcite{mieno2017} point out that these estimates may exhibit bias due to (non-classical) measurement error and/or poor identification. Furthermore, data limitations have restricted previous studies to relatively narrow geographies, which may reduce their external validity.
By contrast, our sample includes thousands of farms throughout California's Central Valley, one of the most productive agricultural regions in the world. Our data also allow us to overcome the standard measurement issues and identification challenges---via detailed technical audits that precisely characterize the electricity-to-groundwater conversion factor, and via exogenous variation in Pacific Gas \& Electric's (PGE) agricultural electricity tariffs.

%In contrast, we estimate the price elasticity of demand for agricultural groundwater throughout California's Central Valley, one of the most important agricultural regions in the world. We use a multi-step estimation process. We begin by using plausibly exogenous variation in energy pricing, both in the cross-section and over time to calculate the price elasticity of electricity consumption. We then combine this with detailed (time-varying) data on farm-specific pump characteristics and spatially explicit information on groundwater levels. With these additional data, we can estimate the price elasticity of groundwater demand with respect to (1) electricity prices, (2) groundwater levels, and finally, (3) water costs. 

\begin{comment}

\begin{itemize}

\item APEP pump tests are the other piece of this puzzle, and we can observe pump-specific conversion rates.

\end{itemize}
	
\end{comment}

