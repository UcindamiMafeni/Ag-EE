%
% BACKGORUND 
%

\subsection{Agriculture in California}
California is a major player in global agricultural production, and its nearly \$33 billion in crop value in 2018 represented 18 percent of the U.S. total (\textcite{ers2020}). California's 77,000 farms produce over 400 commodities, including more than half of all fruits, nuts, and vegetables grown in the United States. In fact, California is the sole domestic producer of many high-value crops, including almonds, artichokes, olives, and walnuts (\textcite{cdfa2011}).  



%\subsubsection{Agricultural water use}
Water is an essential input for California's agricultural production. Nearly 80 percent of the state's annual water consumption occurs in the agricultural sector, where crop irrigation is the primary end use. California has nearly 8.3 million harvested acres of cropland, 7.9 million of which are irrigated (\textcite{crs2015}). Many of California's crops require large amounts of water. For example, hay, almonds, grapes, and rice---four of California's top crops by acreage---all require at least 3 acre-feet per acre per year, with rice using 5 acre-feet per acre per year (\textcite{bruno2019}.\footnote{The average California household uses 0.52 acre-feet per year (\textcite{hanak2011}.} At the same time, droughts of increasing severity have raised serious concerns about the (over)use of water for agriculture in California.

A simple time-series analysis provides suggestive evidence that drought is associated with substantial cropping changes: Figure~\ref{fig:acreage_bars2017} shows that after the 2011--2016 drought, farmers had substantially reduced land in water-intensive but relatively low-value crops such as alfalfa and winter wheat, substituting similarly water-intensive but high-value crops, such as almonds and grapes. In order to water these thirsty crops, farmers rely on groundwater and/or surface water---two water sources with significantly different governance structures (\textcite{sawyers2007}).
\paragraph{Surface water}
Approximately 40 percent of California's surface water is used in agriculture. 61 percent of irrigation water comes from surface sources, with groundwater making up the remaining 39 percent (\textcite{cdwr2015}). Surface water rights in California follow strict rules. Most farms with access to surface water obtain it via irrigation districts.\footnote{Irrigation districts were established between 1860 and 1950, and their boundaries have remained essentially fixed over time. Though some individual farms do have their own water entitlements, the vast majority of these allocations belong to districts. These agricultural cooperatives divert water from large rivers and canals, and distribute this water to farmers. Individual farmers receiving water proportional to their acreage within the district (\textcite{schlenker2007}). \textcite{hagerty2019} provides a detailed description of surface water rights in California.} In addition to obtaining surface water from individual rights or irrigation districts, farmers have a limited ability to purchase water on the open market. However, these trades constitute only a very small share of total water deliveries, and the prices are extremely high (\textcite{hagerty2018}).

\begin{comment}

Irrigation districts obtain water through two main sources: project contracts and water rights. Project contracts allow irrigation districts to withdraw water from state- and federally-operated canals.\footnote{There are three such water projects in California: the State Water Project, the Central Valley Project, and the Lower Colorado Project. The State Water Project is managed by the California government; the other two are federally managed.} Water rights allow irrigation districts to withdraw water from nearby rivers and streams. In California, These rights are under a prior appropriation system, meaning that rights are established by the first claimant, and held as long as water is continuously used.  

Both water rights and project contracts provide irrigation districts with fixed \emph{maximum} appropriations over time, but annual allocations can vary across years. Under drought conditions, water rights are allocated according to the age of the right. Senior rightsholders, with rights established prior to 1914, are entitled to their full allocation in all years. Junior rightsholders, with rights established in and after 1914, are ranked in order of age of the right, and allocated water until the maximum allotment has been distributed. {\color{blue} WE SHOULD TRIPLE CHECK TO MAKE SURE THIS IS ACTUALLY ACCURATE.} Project contracts also vary their allocations year by year, using an algorithm which depends only on environmental conditions. {\color{blue} something about how this works by giving you a share of your max allotment.} Information on allocations is revealed to farmers well in advance of the growing season.

\end{comment}


\paragraph{Groundwater}
In normal weather conditions, groundwater supplies 30 to 40 percent of all water end uses in California. However, this rises to close to 60 percent in drought years, when surface water is unusually scarce (\textcite{cdwr2014}).
In contrast to the strictly defined surface water rights, agricultural groundwater rights in California tend to be far more vague. The typical groundwater right is ``overlying,'' meaning that landowners whose property sits above an aquifer have the right to extract the underlying groundwater.\footnote{
There are also ``appropriative'' groundwater rights, for users who do not own land above the aquifer. These rights are lower-priority than the overlying rights, and users may only exercise appropriative rights in the case of a surplus.}
The vast majority of groundwater use is unmetered, and users face no variable costs of extraction beyond the energy costs of pumping (\textcite{bruno2018}).\footnote{There are limited exceptions to this rule: a few irrigation districts impose a per-unit price on groundwater, but this remains rare (\textcite{bruno2018}).} Hence, a farmer may extract as much groundwater as he chooses, conditional on owning the overlying property rights. 

Many of California's groundwater basins are ``overdrafted,'' meaning that withdrawals exceed the pace of replenishment. As of 2017, some agricultural regions faced overdraft of 2 million acre-feet annually. This has led to a substantial decline in groundwater levels in the Central Valley, most notably in the Tulare and San Joaquin groundwater basins---which have lost a combined 135 million acre-feet of groundwater since 1925.\footnote{See: \url{https://www.ppic.org/publication/groundwater-in-california/}} The state faced a severe drought in 2014, with groundwater levels reaching historic lows in many portions of the state. 21 of the state's 515 groundwater basins are now considered ``critically overdrafted.''

In September 2014, California lawmakers responded to drought conditions by passing the Sustainable Groundwater Management Act (SGMA). This sweeping groundwater legislation is the first statewide regulatory effort to mitigate over-extraction of groundwater. SGMA comprises three separate bills. AB 1739 empowers California's Department of Water Resources (DWR) or local groundwater sustainability agencies (GSAs) to charge fees for groundwater extraction, and requires GSAs to prepare groundwater sustainability plans (GSPs). SB 1319 authorizes GSAs to implement these GSPs. SB 1168 mandates that groundwater end uses be both reasonable and beneficial, and enables GSAs and the DWR to require groundwater monitoring. 

This legislation represents the future of groundwater management in California, with the goal of achieving long-run sustainability by 2042. In order to meet these sustainability targets, some basins will have to reduce groundwater use by between 20 and 50 percent. Farmers are expected to meet these targets with a combination of reducing irrigation intensity and/or technology adoption, water trade with urban areas, and land fallowing or shifts towards less water-intensive crops (\textcite{bruno2019}). However, SGMA's GSPs may not begin to bind for many years, leaving groundwater pumping effectively unregulated in the interim.

\subsection{Electricity for pumping}
Electricity is an essential input to groundwater pumping. The California Energy Commission reports that water use accounts for 19 percent of California's electricity consumption, and close to 8 percent of the state's energy is used on farms (\textcite{cec2005}). The state's investor-owned utilities invest nearly \$50 million annually in agricultural energy efficiency. This makes water use---in particular, agricultural water use---a key component of California's energy policy goals.

To estimate groundwater demand, we exploit the fact that electricity is a major determinant of pumping costs. Several previous papers have used variation in energy costs to estimate the price elasticity of groundwater demand (\textcite{hendricks2012}; \textcite{pfeiffer2014}; \textcite{badiani2015}; and \textcite{mieno2017}). \textcite{mieno2017} point out that these estimates may exhibit bias due to (non-classical) measurement error and/or poor identification. Furthermore, data limitations have restricted previous studies to relatively narrow geographies.

We build on this existing literature with a sample that covers thousands of farms throughout California's Central Valley, one of the most productive agricultural regions in the world. We are able to overcome the standard measurement issues in groundwater demand estimation via detailed technical audits that precisely characterize pump-specific electricity-to-groundwater conversion factors. Using exogenous variation from Pacific Gas \& Electric's (PGE) agricultural electricity tariffs, we are also able to overcome the standard identification challenges in this literature.

