%
% INTRO 
%
Groundwater is an essential input into agricultural production around the globe, responsible for supplying water to 38 percent of irrigated acre worldwide (\textcite{siebert2010}). However, recent scientific evidence documents rapid draw-downs of global aquifers, with key agricultural areas seeing the water table fall by over 4cm per year.\footnote{\url{https://www.nytimes.com/interactive/2019/08/06/climate/world-water-stress.html}} Given that global climate change is projected to increase the frequency and severity of droughts (\textcite{famiglietti2014}), governments will face a greater urgency to enact policies to manage common-pool groundwater resources. For agricultural producers who rely on groundwater for irrigation, both groundwater scarcity itself and groundwater management policies increase the costs of growing crops.

In this paper, we generate novel empirical estimates of farmers' response to changes in groundwater pumping costs in California, one of the world's most valuable crop-producing regions. California produces 17 percent of total U.S.\ crop value, and its farmers rely heavily on groundwater for irrigation. Despite rapidly declining aquifer levels and a series of severe droughts, groundwater extraction remains largely unregulated in California, and most farmers face no meaningful restrictions on pumping. The state is currently implementing its Sustainable Groundwater Management Act, which will introduce sweeping regulations on groundwater use. %This underscores the importance in predicting both the extent to which farmers will respond to new pumping regulations, and their means of adapting to higher irrigation costs.
The effectiveness and economic consequences of any such groundwater regulation depends on both the extent to which farmers will respond to new pumping regulations, and their means of adapting to higher irrigation costs.

We begin by estimating the price elasticity of demand for agricultural groundwater. %We use existing work for (1), including (\textcite{sunding2010}).\footnote{In this draft, we make simplifying assumptions for these hydrological estimates. Future work will expand this to include more realistic hydrological assumptions.} 
Despite the importance of groundwater as an agricultural input, estimating this elasticity has historically proven difficult, in large  part because groundwater use is typically neither priced nor measured. 
We overcome these challenges by leveraging the fact that electricity is the main variable input in groundwater extraction. Given data on electricity prices and quantities, along with  pump-specific mappings from energy input to groundwater output, we are able to construct accurate measures of groundwater prices and quantities. We assemble a novel dataset that combines (i) confidential electricity consumption data for all agricultural customers served by Pacific Gas \& Electric (PGE), California's largest electric utility; (ii) technical pump efficiency audits for nearly 12,000 groundwater extraction points; (iii) publicly available groundwater measurements across space and time, for all major California aquifers; and (iv) satellite-derived crop type measurements.
Using exogenous variation both in PGE's electricity tariff schedules and in average groundwater levels, we identify demand elasticities for both electricity and groundwater.

We estimate farmers' price elasticity of demand for electricity to be $-1.17$, which is much more elastic than prior estimates of electricity demand in the residential and commercial/industrial sectors. Next, we estimate farmers' price elasticity of demand for groundwater, where we separately identify the effect of groundwater price changes coming from variation in electricity prices vs.\ variation in groundwater depths. We recover nearly identical groundwater demand elasticities: $-1.39$ for electricity-induced price changes vs.\ $-1.37$ for depth-induced price changes. These statistically indistinguishable estimates suggest that farmers are equally attentive and responsive to either source of variation in groundwater pumping costs, consistent with the predictions of standard Neoclassical theory. We also estimate a single elasticity of demand for groundwater of $-1.12$, identified using \emph{only} changes in PGE's agricultural electricity tariffs. These estimates are again much more elastic than previous groundwater demand estimates from the existing literature.% likely due to farmers' ability to {\color{magenta}switch between crops with different water requirements and to fallow land when the cost of water makes crop production unprofitable.}
%substitute between groundwater and surface water.\footnote{In ongoing work, we are incorporating data on farmers' surface water availability in order to estimate the elasticity of substitution between water sources.}


Next, we explore the mechanisms behind the large elasticities we estimate. 
%We test two primary hypotheses: first, that farmers are substituting between ground and surface water; and second, that farmers are switching between crops when groundwater prices rise. {\color{blue} something on how we do this; something on what we find?}
We consider four possible mechanisms: (i) applying less water to existing crops; (ii) changing irrigation efficiency; (iii) switching water sources; and (iv) switching crops or fallowing. We are able to rule out (i) and (ii), and we develop a stylized model of farm crop choice and irrigation costs to provide theoretical guidance on (iii) and (iv). This model shows that (iii) is only likely at extremely high groundwater pumping costs, at which point farmers may substitute groundwater for water purchased on the open market (\textcite{hagerty2018}). On the other hand, detailed crop budget studies suggest that crop switching and/or fallowing are likely to occur in the range of groundwater prices we calculate from PGE data. This leaves crop switching as the leading candidate mechanism driving farmers' groundwater demand response.

We conduct several empirical tests to provide evidence for crop switching as the primary mechanism. We estimate annual elasticities that are similar to our monthly elasticities, suggesting that farmers are not arbitraging water sources within a growing season. We also find substantial annual elasticities on both the intensive and extensive margins, which is consistent with farmers switching to less-water-intensive crops and fallowing land. We then estimate heterogeneous elasticities and find the largest responsiveness for farms that switch between annual and perennial crops during our sample period, providing additional support for crop switching. Finally, we directly test for crop switching and find suggestive evidence that farmers switch from annual crops to either perennials or fallowing in response to higher prices of electricity or groundwater. 


This paper makes three main contributions. First, we provide the first large-scale estimates of electricity demand in a major energy-using sector: agriculture in California, one of the most important agricultural sectors in the world.
While many studies estimate the relationship between electricity prices and consumption in the residential sector (\textcite{alberini2011}; \textcite{fell2014}; \textcite{ito2014}; \textcite{deryugina2018}), far fewer have focused on commercial/industrial electricity consumption (\textcite{paul2009}; \textcite{jessoe2015}; \textcite{blonz2016}). To the best of our knowledge, there exists no comparable study estimating the price elasticity of electricity demand in the agricultural sector. By leveraging microdata for thousands of agricultural consumers across PGE's service territory, along with plausibly exogenous changes in farmers' marginal electricity prices, we identify California farmers as relatively elastic electricity consumers.

Second, we estimate the elasticity of groundwater demand for California farmers---a policy-relevant elasticity that has proven elusive due to both data and identification challenges (\textcite{mieno2017}). Our empirical strategy overcomes many of these challenges by combining comprehensive electricity consumption data with technical audits of groundwater pumps, and by leveraging exogenous variation in electricity prices (a major component of pumping costs) to credibly identify changes in farmers' effective price of groundwater. 
Many previous studies have estimated water demand outside the agricultural sector (\textcite{hewitt1995}; \textcite{renwick2000}; \textcite{olmstead2007}), while others have focused specifically on groundwater demand in agriculture (\textcite{hendricks2012}; \textcite{pfeiffer2014}; \textcite{badiani2015}). We provide well-identified groundwater demand estimates, for thousands of farms from one the most important agricultural regions in the world: California's Central Valley.

Finally, we go beyond simply estimating how farmers respond to groundwater cost increases by providing evidence of \emph{how} they reoptimize in the face of rising input costs. Given that farmers appear to respond Neoclassically to cost increases (either higher electricity prices or greater groundwater depths), our demand estimates imply that they are likely to respond similarly to cost increases associated with groundwater management policies. We show that crop switching (and fallowing) is the primary mechanism underlying our large elasticity estimates. This suggests that price-based groundwater policies could lead to large shifts in crop choice, which would have major welfare implications for land use and agricultural markets. 





This paper proceeds as follows. Section~\ref{sec:background} provides background on groundwater pumping, California agriculture, and energy use in farming. Sections~\ref{sec:data} and \ref{sec:empirics} describe our data and empirical strategy. Section~\ref{sec:results} presents our demand elasticity estimates. We present a simple theoretical model of on-farm water use and test for mechanisms in Section~\ref{sec:mechanisms}. Section~\ref{sec:conclusion} concludes.
