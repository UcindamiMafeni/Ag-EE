%
% INTRO 
%
Groundwater is an essential input into agricultural production, responsible for supplying water to 38 percent of irrigated acres worldwide (\textcite{siebert2010}). Recent scientific evidence documents rapid drawdowns of global aquifers, with key agricultural areas seeing their water tables fall by over 4 cm per year.\footnote{\url{https://www.nytimes.com/interactive/2019/08/06/climate/world-water-stress.html}} Given that global climate change is projected to increase the frequency and severity of droughts (\textcite{famiglietti2014}), governments will face a greater urgency to enact policies to manage common-pool groundwater resources. For agricultural producers who rely on groundwater for irrigation, both groundwater scarcity itself and groundwater management policies increase the costs of growing crops.

In this paper, we generate novel empirical estimates of farmers' response to changes in groundwater pumping costs in California, one of the world's most valuable agricultural regions. California produces 18 percent of total U.S.\ crop value, and its farmers rely heavily on groundwater for irrigation. Despite rapidly declining aquifer levels and a series of severe droughts, most California farmers face no meaningful restrictions on groundwater extraction. 
%remains largely unregulated in California, and most farmers face no meaningful restrictions on pumping. 
The state is currently implementing the Sustainable Groundwater Management Act (SGMA), California's first legislation governing groundwater use at scale. %This underscores the importance in predicting both the extent to which farmers will respond to new pumping regulations, and their means of adapting to higher irrigation costs.
The effectiveness and economic consequences of any such groundwater regulation depend on both the extent to which farmers will respond to new pumping regulations and their means of adapting to higher irrigation costs.

We begin by estimating the price elasticity of demand for agricultural groundwater. %We use existing work for (1), including (\textcite{sunding2010}).\footnote{In this draft, we make simplifying assumptions for these hydrological estimates. Future work will expand this to include more realistic hydrological assumptions.} 
%Despite the importance of groundwater as an agricultural input, e
Estimating this elasticity has historically proven difficult, in large  part because groundwater is typically neither priced nor measured. 
We overcome these challenges by leveraging the fact that electricity is the main variable input in groundwater extraction. Using data on electricity prices and quantities, along with farm-specific mappings from energy inputs to groundwater extraction volumes, we are able to construct measures of groundwater prices and quantities. We assemble a novel dataset that combines (i) confidential electricity consumption data for all agricultural customers served by Pacific Gas \& Electric (PGE), California's largest electric utility; (ii) technical pump efficiency audits for nearly 12,000 groundwater extraction points; and (iii) publicly available groundwater measurements across space and time, for all major California aquifers.%\footnote{\color{red} I wanted to hold off on the crop types for a bit, since they aren't part of the elasticity estimates} 
%; and (iv) satellite-derived crop type measurements.

Using exogenous variation in PGE's electricity tariff schedules, we causally identify demand elasticities for both electricity and groundwater. We estimate farmers' price elasticity of demand for electricity to be $-1.17$, which is much more elastic than prior estimates of electricity demand in the residential and commercial/industrial sectors. For groundwater, we estimate a demand elasticity of $-1.12$. This is much more elastic that previous estimates from the literature on agricultural groundwater demand. Both elasticity estimates are robust to a variety of sensitivities, and together imply that California's forthcoming groundwater management efforts have the potential to yield the substantial reductions in extraction necessary to meet the state's sustainability goals.


Given these large elasticities, a natural question is: \emph{how} are farmers reducing their water use in response to increases in groundwater costs? We consider four potential mechanisms that farmers may be engaging in: (i) applying less water to existing crops; (ii) changing irrigation efficiency; (iii) switching water sources; and (iv) switching crops or fallowing. We present empirical evidence against (i): farmers do not appear to adjust water consumption in response to large short-run fluctuations in price. We can also rule out (ii): we estimate quantitatively similar groundwater elasticities in the short windows around pump tests, where major changes in pumping capital are highly unlikely. We then develop a stylized model of farm crop choice and irrigation costs to provide theoretical guidance on (iii) and (iv). This model reveals that (iii) is only likely at extremely high groundwater pumping costs---which we rarely observe in our data---at which point farmers may substitute groundwater for water purchased on the open market (\textcite{hagerty2018}).

This leaves crop switching as the leading candidate mechanism driving farmers' groundwater large demand response. We conduct several empirical tests to provide evidence in support of this crop switching mechanism. We estimate annual elasticities that are similar to our monthly elasticities, suggesting that farmers are not arbitraging water sources within a growing season. We find that these annual elasticities can be decomposed into substantial intensive- and extensive-margin responses to increasing groundwater costs, consistent with farmers switching to less-water-intensive crops and fallowing land. 

Building on these results, we estimate a discrete choice model of farmers' cropping decisions to directly quantify the causal effect of groundwater costs on land use. Using a multinomial probit model, identified using exogenous variation in PGE's electricity tariffs, we find that increases in groundwater costs reallocate cropland across four broad crop types. A 10 percent increase in the cost of groundwater causes farmers to increase the proportion of land in fruit and nut perennials by 1.0 percentage point, increase fallowed land by 0.4 percentage points (almost identical to our extensive-margin semi-elasticity), decrease land in annuals by 0.9 percentage points and reduce land in other perennials by 0.5 percentage points. Because we use a static discrete choice model, these are likely underestimates of the true impacts of increasing groundwater costs on crop reallocation (\textcite{scott2013}). %Dynamics are likely important in this context, however, meaning that these static estimates understate true switching. In ongoing work, we are adapting the dynamic discrete choice framework developed by \textcite{scott2013} to our context, in order to incorporate state dependence in crop choice.


Finally, we simulate how cropland would be reallocated in response to a counterfactual groundwater tax---a regulation that could internalize open-access externalities and achieve sustainable levels of groundwater extraction.\footnote{The water economics literature typically cites two main externalities associated with groundwater extraction (\textcite{provencher1993}). The first is the ``stock externality'', which arises from the finite nature of non-renewable groundwater stocks, driving farmers to collectively extract faster than the social planner's optimal extraction path. The second is the ``pumping cost'' externality: when a farmer extracts an acre-foot of groundwater from an aquifer, the water level falls, increasing pumping costs for other (nearby) users. Recently, other potential externalities such as air quality issues associated with soil drying and land subsidence have been raised as well.}  Under SGMA, California will require high-use basins to reduce extraction by between 20 and 50 percent (\textcite{bruno2019}). We find that a \$10 per acre-foot tax on groundwater, representing approximately a 25 percent increase in pumping costs, would result in a 27 percent reduction in groundwater extraction, near the low end of this targeted curtailment range. We estimate that this same tax would lead farmers to reallocate nearly 8 percent of cropland to a different crop type. While the level of the efficient Pigouvian groundwater tax remains unknown, if the true externality is around \$10 per acre-foot, our results imply that that 8 percent of total cropland in California is currently misallocated from its socially optimal use.


This paper makes three main academic contributions. First, we provide the first large-scale empirical estimates of the impact of groundwater pricing on crop choice in one of the most valuable agricultural regions of the world: California. We demonstrate that, as groundwater extraction costs rise, farmers respond by increasing fallowing, shifting into fruit/nut perennials (among the highest-revenue crops per acre-foot of water), and shifting out of low-value perennials and annual crops. These results have important implications for agricultural markets, as California is a monopoly producer of many crops that may become less prevalent under groundwater regulation. They also contribute estimates from an understudied sector to the broader literature on the impacts of environmental regulation.



Second, to quantify the impacts of groundwater cost increases on land use, we begin by contributing novel estimates of the price elasticity of electricity demand in agriculture---a major, but heretofore overlooked, electricity end-use sector. While many studies estimate the relationship between electricity prices and consumption in the residential sector (\textcite{alberini2011}; \textcite{fell2014}; \textcite{ito2014}; \textcite{deryugina2018}), far fewer have focused on commercial/industrial electricity consumption (\textcite{paul2009}; \textcite{jessoe2015}; \textcite{blonz2016}). To the best of our knowledge, there exists no comparable study estimating the price elasticity of electricity demand in the agricultural sector. By leveraging microdata for thousands of agricultural consumers across PGE's service territory, along with plausibly exogenous changes in farmers' marginal electricity prices, we identify California farmers as unusually elastic electricity consumers (a central elasticity estimate of $-1.17$), with potential implications for agricultural electricity pricing. 

Third, we leverage the central role of electricity use in groundwater extraction to estimate the elasticity of groundwater demand for California farmers---a policy-relevant elasticity that has proven elusive due to both data and identification challenges (\textcite{mieno2017}). Our empirical strategy overcomes many of these challenges by combining comprehensive electricity consumption data with technical audits of groundwater pumps, and by leveraging exogenous variation in electricity prices (a major component of pumping costs) to credibly identify changes in farmers' effective price of groundwater. 
Many previous studies have estimated water demand outside the agricultural sector (\textcite{hewitt1995}; \textcite{renwick2000}; \textcite{olmstead2007}), while others have focused specifically on groundwater demand in agriculture (\textcite{hendricks2012}; \textcite{pfeiffer2014}; \textcite{badiani2015}). We provide well-identified groundwater demand estimates for thousands of farms in one the most important agricultural regions in the world: California's Central Valley. We find that farmers are surprisingly elastic, with a central estimate of $-1.12$. We rule out several hypotheses about what might be driving this large elasticity, and demonstrate that crop reallocation is the major mechanism behind farmers' responses to increases in water costs.  

%Finally, we go beyond documenting that farmers are price-responsive groundwater consumers by {\color{blue} demonstrating that crop switching and fallowing is the primary mechanism by which they re-optimize in the face of rising input costs.} Our results imply that groundwater policies that raise the (effective) cost of water will likely lead to large shifts in crop choice. 

Beyond our contributions to the academic  literature, our findings have direct and immediate policy relevance for California agriculture, as SGMA goes into effect. If we extrapolate beyond our sample to all agriculture in California agriculture, our simulations suggest that the \$10 per acre-foot tax required to meet the lower end of SGMA's groundwater sustainability targets would reallocate nearly two million acres of cropland across the state. Our elasticity estimates also suggest that meeting the approximately 50 percent groundwater use reduction targets in severely overdrawn basins will require a tax greater than \$15 per acre-foot---more than 37 percent of the average cost per acre-foot we observe in our sample. 

This paper proceeds as follows. Section~\ref{sec:background} provides background on groundwater pumping, California agriculture, and energy use in farming. Sections~\ref{sec:data} and \ref{sec:empirics} describe our data and empirical strategy. Section~\ref{sec:results} presents our demand elasticity estimates. In Section~\ref{sec:mechanisms}, we explore mechanisms, and demonstrate the impacts of groundwater costs on crop switching.
Section~\ref{sec:conclusion} concludes. 
