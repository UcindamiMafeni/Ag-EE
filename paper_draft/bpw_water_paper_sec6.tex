%
% CROP SWITCHING MECHANISM
%
We now explore possible mechanisms underlying the large elasticities we find in Section~\ref{sec:results}. When we observe a farmer consume less electricity at a pump with a dedicated electricity meter, four broad mechanisms could explain this decrease:
\begin{enumerate}
	\item The farmer applies less irrigation water, without changing crop types.
	\item The pump's kWh-to-AF groundwater production function changes.
	\item The farmer switches to an alternate water source.
	\item The farmer switches crops or fallows the land. 
\end{enumerate}

\paragraph{Applying less water, without changing crops}
We should expect this first mechanism to manifest in the short run, as a response to high-frequency cost shocks. To test for this, we exploit additional richness in our PGE electricity data. Beyond monthly billing data, we also observe 336 million observations of hourly electricity consumption for 10,930 service points in our estimation sample. All of these farmers are on time-varying electricity tariffs: during summer months, their average marginal electricity price nearly doubles during 12--6pm peak periods. Figure~\ref{fig:hourly_consumption_vs_price} plots these average marginal prices against the density of their hourly electricity consumption during summer months. We observe no pronounced decrease in consumption in response to these large swings in marginal price. This suggests that short-run adjustments to irrigating existing crops are not the primary mechanism driving our elasticity estimates.\footnote{In ongoing work, we are extending our econometric analysis to formally estimate short-run demand elasticities at the daily and hourly levels.}This is also consistent with our informal conversations with California growers, who tend to adhere to crop-specific irrigation targets.\footnote{Accelerated orchard management would be another manifestation of this mechanism, which we hope to explore further in future work. Farmers may respond to pumping cost shocks by shifting forward the timing of when they replace old trees with young saplings, which require less water as they mature.}


\paragraph{Changing the groundwater production function}
By contrast, changes in the kWh-to-AF function should manifest over longer time scales, as a result of pump depreciation, maintenance, and upgrades. To test whether pumping technology changes are driving our elasticity estimates, or whether measurement error in $\widehat{\frac{\text{kWh}}{\text{AF}}}_{it}$ from infrequent pump tests are biasing our results, we re-estimate Equation (\ref{eq:reg_water_combined}) using only observations that are close in time to a pump test that we observe. Table~\ref{tab:water_months_from_pump_test} presents the results, which are quite stable even when we restrict- the sample to the 30 percent of observations within 12 months of a pump test. This implies that unobserved changes in kWh-to-AF are unlikely to be impacting our results.\footnote{Another possibility is that unobserved changes in other irrigation capital impact the share of pumped water than ultimately reaches crops. While we lack data on irrigation capital ``downstream'' from pumps, it is reassuring that we find similar elasticities using a significantly shortened time period for estimation.}


 % {\color{blue}To test whether farmers investing in new pumping technology is impacting our results, we re-estimate Equation (\ref{eq:reg_water_combined}) using only observations that are close in time to our observed pump tests. If farmers were responding to groundwater costs by changing their technology, we would expect the elasticity around pump tests to be smaller than our preferred estimate of $-1.12$. Table~\ref{tab:water_months_from_pump_test} presents the results. Even when we restrict the sample to be within 12 months of a pump test, we find an elasticity of $-0.90$.} As our results do not meaningfully change when we restrict our sample to observations close to the timing of pump $i$'s pump test, this mechanism is unlikely to be driving our results. 

\subsection{A stylized model of farmer decision-making}
\label{sec:farmmodel}
This leaves two candidate mechanisms: switching water sources and switching crops. Here, we develop a stylized model to characterize the economics underlying both switching margins. Let $i$ index farms, or atomistic pieces of cropland with area $A_i$. Farm $i$ makes a discrete choice to plant crop $k$ from a set of potential crops $\mathcal K$, with an outside option of fallowing ($k=0$). The choice of crop determines \emph{ex ante} expected yields $Y_i(k)$, water required for irrigation $W_i(k)$, and  non-irrigation costs $F_i(k)$. Yields, water, and other costs vary cross-sectionally within crops, due to  heterogeneous climate, soil quality, irrigation capital, etc.\footnote{
For simplicity, we abstract away input re-optimization within crops. This focuses our model solely on the planting margin, while also aligning it with standard crop budgeting calculations farmers typically use to make such \emph{ex ante} planting decisions. This stylized static model also ignores the obvious state-dependence inherent in choosing perennial (or annual) crops.
}
All farms are price-takers in the output market, facing common crop prices $p^k$. Farm $i$ chooses the crop $k$ that maximizes its profits:
\begin{equation}
\max_{k \in \mathcal K} 	~\pi_i(k) ~=~ p^k Y_i(k) ~-~C_i\big(W_i(k)\big) ~-~ F_i(k) \label{eq:profit_max_choice} 
\end{equation}


Irrigation costs $C_i(W)$ are farm-specific and weakly convex in $W$, since farmers may irrigate using surface water allocated by their water district, pumped groundwater, and/or water purchased on the open market. Water districts typically have the lowest cost per acre-foot, but limited allocations may not be sufficient to meet irrigation needs, especially in drought years. This may force farmers to pump their own groundwater, with costs per acre-foot that are (typically) higher and may increase convexly in the quantity extracted. Open market water prices tend to be much higher than both district-allocated surface water and pumped groundwater, due to high costs of physically moving water. However, district allocations and pumping costs are sufficiently heterogeneous that some California farmers rely on water markets for irrigation (\textcite{hagerty2018}). 

The upper left panel of Figure \ref{fig:water_cost_cartoons} depicts a hypothetical irrigation price schedule for a farm that irrigates with both surface water (allocated by its irrigation district) and pumped groundwater. This representative farm $i$ has chosen crop $k^0$, and the shaded region under the price function depicts its total cost of irrigating, $C_i\big(W_i(k^0)\big)$. If the farm experiences a pumping cost shock due to either an electricity price increase or a groundwater depth increase, the groundwater piece of its price schedule will shift up. The upper right panel of Figure \ref{fig:water_cost_cartoons} illustrates how such a pumping cost shock would increase farm $i$'s cost of irrigating crop $k^0$ by the shaded area $\Delta C_i\big(W_i(k^0)\big)$. 

Our econometric results show that such a pumping cost shock (holding the rest of the price schedule constant) causes average groundwater consumption to decrease. The bottom panels of Figure \ref{fig:water_cost_cartoons} illustrate how this consumption decrease could come through either crop switching or water source substitution. In the lower left panel, farm $i$ switches to a less water-intensive crop $k^1$, thereby reducing both its groundwater consumption and its total water consumption. In the lower right panel,  a larger  pumping cost shock causes farm $i$ to switch to the open market backstop; while it continues to consume $W_i(k^0)$ acre-feet of water, new water purchases now crowd out its groundwater consumption. As long as district-allocated surface water is inframarginal for groundwater users (\textcite{hagerty2019}), water source substitution is only likely to occur at extremely high prices. \textcite{hagerty2018} reports the distribution of prices for 671 California water transactions, with a mean price of \$221 per acre-foot. Since the PGE data almost never imply pumping costs above \$200 per acre-foot, and since California water markets are relatively thin, crop switching (and fallowing) appears more likely as a mechanism for groundwater demand response.


%We use detailed crop budget studies produced by the University of California, Davis to explore the extent to which higher pumping costs may push farmers to either switch crops or fallow. The ``Davis Cost Studies'' report detailed revenue and cost data for test plots of specific crops across various regions of California.\footnote{Current cost studies are available at \url{https://coststudies.ucdavis.edu/en/current/}.}
%We digitized budget line items for 88 studies covering 65 unique crops, allowing us to compare profitability across crops for a range of hypothetical water prices. Figure \ref{fig:davis_line_crossing} plots predicted profits per acre from 5 individual crop studies that are all specific to the Southern San Joaqu\'{i}n Valley. Several stylized patterns emerge from this figure. First, as the water price increases, profits fall faster for more water-intensive crops. Second, within both tree and field crops, there are switching points at which a small increase in the average price per acre-foot may cause a farmer to switch to a less water-intensive crop.\footnote{
%Since crop choice is discrete, the relevant water price both in the Davis Cost Studies and in our stylized model is the average price per acre-foot, \emph{not} the marginal price per acre-foot. This implies that inframarginal surface water allocations will dampen the impact of groundwater cost shocks on crop switching.
%}
%Third, for field crops, increasing water prices may lead to negative profits before or after a given switching point, which may cause a farmer to fallow. Fourth, the hypothetical groundwater prices where switching or fallowing might occur are far below \$221 per acre-foot, the average price on the open water market. While these comparisons are only illustrative and numerous data caveats apply,\footnote{For example, Figure \ref{fig:davis_line_crossing} compares cost studies from different years, without adjusting commodity prices. It also ignores dynamics, which are crucial for almond, plum, and alfalfa planting decisions.} they do provide additional support for crop switching and fallowing as the mechanism underlying our groundwater demand estimates.

\subsection{Empirical tests of crop switching}
\label{sec:crop_switching}

\subsubsection{Linear estimates}
We take the model in Section~\ref{sec:farmmodel} to data using several empirical tests. First, we collapse our monthly panel dataset to the service-point-by-year level, and estimate annual elasticities. While monthly data enable us to include more granular fixed effects to non-parametrically control for potential time-varying confounders, annual data more closely align with the timescale of a farmer's cropping choice. We estimate annual elasticities using modified versions of Equations~(\ref{eq:reg_elec}) and (\ref{eq:reg_water_combined}):
\begin{equation}
\sinh^{-1}\big(Q^{\text{elec}}_{iy}\big) = \beta \log\big(P^{\text{elec}}_{iy}\big) + \gamma_{i} + \delta_y + \varepsilon_{iy}
\label{eq:reg_elec_annual}
\end{equation}
and
\begin{equation}
\sinh^{-1}\big(Q^{\text{water}}_{iy}\big) = \beta\log\big({P}^{\text{water}}_{iy}\big) + \gamma_{i} + \delta_y + \varepsilon_{iy} \label{eq:reg_water_annual} 
\end{equation}
where the unit of observation is now the service point-($i$)-by-year-($y$). As in Sections~\ref{sec:empirics_elec}--\ref{sec:empirics_water}, we instrument for the price of electricity (groundwater) with the default within-category price of electricity. Columns (1)--(2) of Table~\ref{tab:elec_water_intens_extens} present the main results, where we estimate an annual demand elasticity of $-0.99$ for electricity and $-0.93$ for groundwater. Since these annual results are similar to our monthly estimates, they suggest that farmers are not simply arbitraging within-year fluctuations in  groundwater prices by switching to surface water.\footnote{Decomposing the variance in monthly $P_{it}^{\text{water}}$ into between-unit, within-unit-within-year, and within-unit-across-year components, we find that 68 percent of the price variation is between units. Of the remaining within-unit variation, 84 percent is within-year, and only 16 percent is across-year. Nevertheless, our monthly and annual elasticity estimates are quite similar, suggesting that the former are mostly driven by within-unit-across-year variation in groundwater price. This provides further evidence against (i) farmers applying less water to existing crops and (ii) short-run substitutions between surface water and groundwater.}



To further investigate crop switching as a mechanism, we estimate pumping changes on the intensive (reducing the amount of non-zero pumping) vs.\ extensive (stopping pumping entirely) margins. First, we estimate Equation (\ref{eq:reg_elec_annual}) for the subset of service points with non-zero groundwater use in each sample year. Next, we estimate a semi-elasticity, by replacing the dependent variable in Equation (\ref{eq:reg_elec_annual}) with $1\big[Q^{\text{water}}_{iy}\big]$, a binary indicator for whether a unit consumes groundwater in a given year. Table~\ref{tab:elec_water_intens_extens} reports these results. In Column (3), we find that the intensive margin elasticity ($-0.22$) is substantially lower than the average elasticity.\footnote{This is consistent with Appendix Table~\ref{tab:elec_regs_ihs_logs}, where we compare the inverse hyperbolic sine (IHS) and log transformations at the monthly level. We find substantially larger elasticities with the IHS (which admits zeros) than with the log (which does not).}
 
By contrast, Column (4) reports a semi-elasticity of $-0.04$ for the extensive margin: a 10-percent increase in effective groundwater price increases the probability that a farmer stops pumping entirely by 0.4 percentage points---perhaps via fallowing. Comparing across Columns (2)--(4), we see further evidence that our elasticities are driven by units moving in and out of pumping, which is consistent with the crop-switching mechanism.


\subsubsection{Discrete choice estimates}\label{sec:discrete_choice}
Building on this evidence, we estimate a discrete crop choice model to directly measure the causal effect of  groundwater costs on cropping decisions.\footnote{Appendix~\ref{app:discrete_choice} provides a step-by-step derivation of the following model.} We begin by re-writing the farm's crop choice optimization problem, from Equation (\ref{eq:profit_max_choice}), as:
\begin{equation}\label{eq:discrete_choice}
\max_{k \in \mathcal K} ~\pi_{iy}(k) ~=~ \beta_k p_{iy}^{water} ~+~ \gamma_{cyk} ~+~ \varepsilon_{iyk}
\end{equation}
where farm $i$ in county $c$ in year $y$ chooses among four possible crop types $k$: annuals, fruit and nut perennials, other perennials, and fallow.
%\footnote{For the purposes of estimation, we aggregate the data to the CLU-year, and use this as the unit of observation. This is a natural match with the geographic scope and timing of the crop choice decision.} 
Within these groups, crops exhibit roughly similar cost structures. We decompose profits for farm $i$ into two components: the cost of groundwater and all other costs and revenues. The total cost of groundwater is a linear function of the price of groundwater, $\beta_k p_{iy}^{water}$, which we allow to vary by crop type. We represent all remaining costs and revenue with the set of parameters $\gamma_{cyk}$, which gives the average annual profits---excluding groundwater costs---from growing crop type $k$ in county $c$ in year $y$. We estimate the choice of crop type using a multinomial probit model. As in our previous regressions, the groundwater price is potentially endogenous, so we instrument with the unit's within-category default electricity price to isolate exogenous variation in the price of groundwater.

In this discrete choice model, we continue to use data aggregated to the annual level. In order to assign crops to units, we match each service point to a USDA CLU based on its latitude and longitude, and use CLU as the cross-sectional unit of analysis. As described in Section~\ref{sec:data}, we use the Cropland Data Layer to assign a crop type to each 30-by-30 meter pixel in each CLU. We aggregate these pixels to the CLU level by assigning each CLU a crop type based on the modal type across pixels within  the CLU, for CLUs where the modal crop type covers more than 50 percent of the CLU.\footnote{In this draft, we drop CLUs where the modal crop type covers less than 50 percent of the CLU.}

Table \ref{tab:probit_results} shows the marginal effects of groundwater price on crop choice and the semi-elasticities of crop choice with respect to groundwater price that are implied by our discrete choice estimates, averaged over all units and reported in percentage points. On average, when the price of groundwater increases by 10 percent, a farmer increases the proportion of land in fruit and nut perennials by 1.0 percentage points and increases fallowed land by 0.4 percentage points, and reduces land in annuals by 0.9 percentage points and other perennials by 0.5 percentage points. This semi-elasticity for fallowing is quite close to the semi-elasticity on the extensive margin of pumping in Table \ref{tab:elec_water_intens_extens} ($-0.04$). Importantly, because we estimate a static discrete choice model in a setting where dynamics are inherently important, these semi-elasticity estimates likely understate the true land-use response (\textcite{scott2013}).\footnote{In ongoing work, we are adapting the model developed by \textcite{scott2013} to our setting, in order to account for the dynamics inherent in crop choice.}

\subsubsection{Counterfactual groundwater taxes}
\label{sec:cf_gw_taxes}
We use these discrete choice estimates to simulate the effects of a uniform Pigouvian groundwater tax on cropping decisions. Because the precise size of the groundwater extraction externality is unknown, we consider a range of possible taxes from \$5 per acre-foot to \$15 per acre-foot, which roughly corresponds to a 12--37 percent increase in the average price of groundwater.\footnote{The average annual groundwater price for CLUs in our sample is \$41.00 per acre-foot.} For each tax level, we add the tax to the observed groundwater price faced by each farmer and use the estimated model parameters to calculate the counterfactual choice probability for each crop type.

Table \ref{tab:sim_acres} reports the estimated total acreage for each crop type within our sample  when farmers face different counterfactual groundwater taxes.\footnote{There are 314,884 acres that we can confidently match to APEP units, which we use for estimation.} In line with the semi-elasticity estimates, as the groundwater tax increases, farmers shift acreage into fruit and nut perennials and into fallowing, and they decrease acreage in annuals and other perennials. We find that a relatively large share of total acreage is reallocated to different crop types in response to relatively moderate groundwater taxes. For a \$10 per acre-foot tax, more than 24,000 acres---nearly 8 percent of total cropland in our sample---are reallocated to a different crop type. Figure~\ref{fig:cf_water_tax} plots how each CLU in our sample would respond to a \$10 groundwater tax. 

If we conducted this counterfactual exercise using a Pigouvian tax equal to true (unknown) marginal external costs at the socially efficient level of groundwater extraction, we could interpret the amount of land reallocated due to the tax as the amount of land that is currently misallocated due to unpriced pumping externalities.
If the true externality were \$10 per acre-foot, our results would imply that 8 percent of cropland in our 314,884-acre sample is currently misallocated.\footnote{We are limited to CLUs in PGE service territory, with PGE service points that match to an APEP pump test. In ongoing work, we are extending the sample to also include agricultural groundwater pumpers in Southern California Edison service territory.} Under the assumption that our sample is representative of all 24.3 million acres of California cropland,
%\footnote{From the USDA's National Agricultural Statistics Service Quick Facts for California: \url{https://www.nass.usda.gov/Quick_Stats/Ag_Overview/stateOverview.php?state=CALIFORNIA}}, 
this result implies that nearly two million acres of cropland are currently growing a socially suboptimal crop type due to groundwater being underpriced. However, to our knowledge there is scant existing evidence on the true external costs from groundwater extraction in California; we encourage future research on this topic. 

Alternatively, we use California policy to benchmark a counterfactual groundwater tax. Under SGMA, California has opted for broad quantity-based groundwater sustainability targets. When the regulation goes into effect, major groundwater basins in California are expected to require groundwater pumping restrictions of 20--50 percent (\textcite{bruno2019}). As shown in the bottom row of Table~\ref{tab:sim_acres}, our groundwater demand elasticity estimates imply that a \$10 per acre-foot tax would yield a 27 percent reduction in groundwater extraction. This is at the lower end of the range of expected curtailment, lending credence to our use of a \$10 tax as our central counterfactual scenario. Regardless of the policy mechanism or stringency, restrictions on California farmers' groundwater use will likely lead to substantial changes in land use.




