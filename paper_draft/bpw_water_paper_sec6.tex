%
% CROP SWITCHING MECHANISM
%
Having established that California farmers are quite responsive to changes in their groundwater pumping costs, we now turn our attention to the mechanisms underlying this demand elasticity. When we observe a farmer consume less electricity at a pump with a dedicated electricity meter, four possible mechanisms could explain this decrease:
\begin{enumerate}
	\item The farmer applies less water to existing crops.
	\item The pump's kWh-to-AF groundwater production function changes.
	\item The farmer switches to an alternate water source.
	\item The farmer switches crops or fallows the land. 
\end{enumerate}
This first mechanism should manifest in the short run, especially for annuals where the cropping decision may not persist across growing seasons. Two results give us confidence that this is not the main mechanism. First, using hourly electricity data from PGE, we find very limited responses of within-day consumption to within-day price changes. Secondly, we find that farmers are substantially less elastic during the summer growing season than during the winter. Both of these results suggest that farmers are not reducing water use on their crops, conditional on cropping decisions.\footnote{These results are available on request.} This implies that small within-crop irrigation adjustments is not the primary mechanism driving our results. By contrast, changes in the kWh-to-AF function should manifest over longer time scales, following pump depreciation, maintenance, and upgrades. As our results do not meaningfully change when we restrict our sample to observations close the timing of pump $i$'s pump test, this mechanism is unlikely to be driving our results. 

\subsection{A stylized model of farmer decision-making}
\label{sec:farmmodel}
This leaves two candidate mechanisms: switching water sources and switching crops. Here, we develop a stylized model to characterize the economics underlying both switching margins. Let $i$ index farms, or atomistic pieces of cropland with area $A_i$. Farm $i$ makes a discrete choice to plant crop $k$ from a set of potential crops $\mathcal K$, with an outside option of fallowing ($k=0$). The choice of crop determines \emph{ex ante} expected yields $Y_i(k)$, water required for irrigation $W_i(k)$, and  non-irrigation costs $F_i(k)$. Yields, water, and other costs vary cross-sectionally within crops, due to  heterogeneous climate, soil quality, irrigation capital, etc.\footnote{
For simplicity, we abstract away input re-optimization within crops. This focuses our model solely on the planting margin, while also aligning it with standard crop budgeting calculations farmers typically use to make such \emph{ex ante} planting decisions. This stylized static model also ignores the obvious state-dependence inherent in choosing perennial (or annual) crops.
}
All farms are price-takers in the output market, facing common crop prices $p^k$. Farm $i$ chooses the crop $k$ that maximizes its profits:
\begin{equation}
\max_{k \in \mathcal K} 	~\pi_i(k) ~=~ p^k Y_i(k) ~-~C_i\big(W_i(k)\big) ~-~ F_i(k)
\end{equation}


Irrigation costs $C_i(W)$ are farm-specific and weakly convex in $W$, since farmers may irrigate using surface water allocated by their water district, pumped groundwater, and/or water purchased on the open market. Water districts typically have the lowest cost per acre-foot, but limited allocations may not be sufficient to meet irrigation needs (especially in drought years). This may force farmers to pump their own groundwater, with costs per acre-foot that are (typically) higher and may increase convexly in the quantity extracted. Open market water prices tend to be much higher than both district-allocated surface water and pumped groundwater, due to high costs of physically moving water. However, district allocations and pumping costs are sufficiently heterogeneous that some California farmers rely on water markets for irrigation (\textcite{hagerty2018}). 

The upper left panel of Figure \ref{fig:water_cost_cartoons} depicts a hypothetical irrigation price schedule for a farm that irrigates with both surface water (allocated by its irrigation district) and pumped groundwater. This representative farm $i$ has chosen crop $k^0$, and the shaded region under the price function depicts its total cost of irrigating, $C_i\big(W_i(k^0)\big)$. If the farm experiences a pumping cost shock due to either an electricity price increase or a groundwater depth increase, the groundwater piece of its price schedule will shift up. The upper right panel of Figure \ref{fig:water_cost_cartoons} illustrates how such a pumping cost shock would increase farm $i$'s cost of irrigating crop $k^0$ by the shaded area $\Delta C_i\big(W_i(k^0)\big)$. 

Our econometric results show that such a pumping cost shock (holding the rest of the price schedule constant) causes average groundwater consumption to decrease. The bottom panels of Figure \ref{fig:water_cost_cartoons} illustrate how this consumption decrease could come through either crop switching or water source substitution. In the lower left panel, farm $i$ switches to a less water-intensive crop $k^1$, thereby reducing both its groundwater consumption and its total water consumption. In the lower right panel,  a larger  pumping cost shock causes farm $i$ to switch to the open market backstop; while it continues to consume $W_i(k^0)$ acre-feet of water, new water purchases now crowd out its groundwater consumption. As long as district-allocated surface water is inframarginal for groundwater users (\textcite{hagerty2019}), water source substitution is only likely to occur at extremely high prices. \textcite{hagerty2018} reports the distribution of prices for 671 California water transactions, with a mean price of \$221 per acre-foot. Since the PGE data almost never imply pumping costs above \$200 per acre-foot, and since California water markets are relatively thin, crop switching (or fallowing) appears far more likely as a mechanism for groundwater demand response.


We use detailed crop budget studies produced by the University of California, Davis to explore the extent to which higher pumping costs may push farmers to either switch crops or fallow. The ``Davis Cost Studies'' report detailed revenue and cost data for test plots of specific crops across various regions of California.\footnote{Current cost studies are available at \url{https://coststudies.ucdavis.edu/en/current/}.}
We digitized budget line items for 88 studies covering 65 unique crops, allowing us to compare profitability across crops for a range of hypothetical water prices. Figure \ref{fig:davis_line_crossing} plots predicted profits per acre from 5 individual crop studies that are all specific to the Southern San Joaqu\'{i}n Valley. Several stylized patterns emerge from this figure. First, as the water price increases, profits fall faster for more water-intensive crops. Second, within both tree and field crops, there are switching points at which a small increase in the \emph{average} price per acre-foot may cause a farmer to switch to a less water-intensive crop.\footnote{
Since crop choice is discrete, the relevant water price both in the Davis Cost Studies and in our stylized model is the average price per acre-foot, \emph{not} the marginal price per acre-foot. This implies that inframarginal surface water allocations will dampen the impact of groundwater cost shocks on crop switching.
}
Third, for field crops, increasing water prices may lead to negative profits before or after a given switching point, which may cause a farmer to fallow. Fourth, the hypothetical groundwater prices where switching or fallowing might occur are far below \$221 per acre-foot, the average price on the open water market. While these comparisons are only illustrative and numerous data caveats apply,\footnote{For example, Figure \ref{fig:davis_line_crossing} compares cost studies from different years, without adjusting commodity prices. It also ignores dynamics, which are crucial for almond, plum, and alfalfa planting decisions.} they do provide additional support for crop switching (or fallowing) as the mechanism underlying our groundwater demand estimates.

\subsection{Empirical tests}
We take the model in Section~\ref{sec:farmmodel} to data using several simple empirical tests. First, we replace our monthly elasticity estimates with annual elasticities. While using monthly data enables us to include granular fixed effects to non-parametrically control for unobserved factors which may impact electricity and groundwater usage, the monthly timescale is somewhat mismatched with the timing of farm decision-making, much of which occurs on an annual timescale. We estimate annual elasticities using modified versions of Equations~(\ref{eq:reg_elec}) and (\ref{eq:reg_water_combined}), on data that we aggregate to the yearly level:
\begin{equation}
\sinh^{-1}\big(Q^{\text{elec}}_{iy}\big) = \beta \log\big(P^{\text{elec}}_{iy}\big) + \gamma_{i} + \delta_y + \varepsilon_{iy}
\label{eq:reg_elec_annual}
\end{equation}
and
\begin{equation}
\sinh^{-1}\big(Q^{\text{water}}_{iy}\big) = \beta\log\big({P}^{\text{water}}_{iy}\big) + \gamma_{i} + \delta_y + \varepsilon_{iy} \label{eq:reg_water_annual} 
\end{equation}
where the unit of observation is now the service point-($i$)-by-year-($y$). As in Sections~\ref{sec:empirics_elec}--\ref{sec:empirics_water}, we instrument for the price of electricity (groundwater) with the default price of electricity. Though there is less identifying variation in this annual model, our first-stage F-statistics remain high: 2286 for electricity and 2065 for water. Columns (1) and (4) of Table~\ref{tab:ann_regs_main} present the main annual results. We find an annual price elasticity of $-1.10$ for electricity and $-0.96$ for groundwater. These annual elasticities being similar to our monthly estimates suggests that farmers are not simply arbitraging between low and high groundwater prices by switching to surface water. 

We then test the extent to which our estimated elasticities are driven by intensive-versus-extensive margin changes in electricity and groundwater use behavior. To test for intensive-margin results, we restrict the sample to only service points that never have a year of electricity bills with zero consumption. To test for extensive-margin results, we replace our dependent variable with an indicator for positive consumption in year $y$. Columns (2) and (5) of Table~\ref{tab:ann_regs_main} present the intensive margin results, and Columns (3) and (6) present the extensive margin results. We find an intensive margin elasticity of demand of $-0.43$ for electricity and $-0.40$ for water. We also find that a 1 percent increase in electricity or groundwater prices increases the likelihood that farmers do not pump by 4 percent. All estimates are statistically significant at the 5 percent level. These results are consistent with a combination of crop switching and fallowing. If water source substitution were the primary mechanism, and small changes in pumping costs induced incremental substitution, we would expect to find an intensive margin elasticity closer to the magnitude of our combined estimates.



Next, we test whether farmers growing different types of crops respond heterogeneously to changes in groundwater costs. For this analysis, we continue to use data aggregated to the annual level. We assign service point $i$ to a Common Land Unit (CLU) based on its latitude and longitude. As described in Section~\ref{sec:data}, we use the Cropland Data Layer to classify each pixel in a CLU as annual, perennial, or fallowed. We then assign crop types to CLUs based on the crop type of the modal pixel, for CLUs where the modal crop type covers more than 50 percent of the CLU. We then classify service points into three categories, which are held fixed throughout our sample. ``Annuals'' are service points in CLUs that are always farming annual crops or are fallowed; ``perennials'' are service points in CLUs that are always farming perennial crops or are fallowed; and ``switchers'' are service points in CLUs that switch between annual and perennial crops during our sample. 

We now estimate Equations~(\ref{eq:reg_elec_annual}) and (\ref{eq:reg_water_annual}) separately for service points in each of these three categories. Table~\ref{tab:ann_regs_het} reports the results of this test. We find that, for both electricity and groundwater, annual-only farmers are much less elastic than perennial-only farmers, who are much less elastic than switchers. For electricity (groundwater), we estimate that annual-only farmers have a demand elasticity of $-0.23$ ($-0.16$), which is not statistically distinguishable from zero. Perennial-only farmers have elasticities of $-1.15$ and $-0.97$, close to our central annual estimates of $-1.10$ and $-0.96$. Switchers are more than twice as elastic as perennial-only farmers, with elasticities of $-2.65$ and $-2.63$. These striking results suggest that farmers that switch between annual and perennial crops (and to a lesser extent, perennial-only farmers) are driving our elasticity estimates. This lends support to the notion that crop switching is an important mechanism underlying our elastic demand estimates.

Finally, we estimate the causal impact of electricity and groundwater pricing on cropping decisions. To do this, we estimate:
\begin{equation}
1[\text{Crop type} = c]_{iy} = \beta\log\big({P}^{\text{elec}}_{iy}\big) + \gamma_{i} + \delta_y + \varepsilon_{iy} \label{eq:reg_crop_switch} 
\end{equation}
and the equivalent for groundwater. The dependent variable, $1[\text{Crop type} = c]_{iy}$ is an indicator, equal to one if service point $i$ has a certain type of landcover, $c \in \{\text{annual, perennial, fallow} \}$ in year $y$, and zero otherwise. We estimate Equation~(\ref{eq:reg_crop_switch}) separately for each of the three crop types. Table~\ref{tab:ann_regs_crop} presents these results. We find some evidence that, as the prices of electricity and groundwater rise, the likelihood that farmers grow annuals declines. These effects are noisy---statistically significant at the 10 percent level---but economically meaningful: for a 1 percent increase in electricity or groundwater price, cropping of annuals falls by 4 percent. We find positive point estimates for perennials (0.02) and fallowing (0.03) for both electricity and groundwater, but these estimates are not significant at the 10 percent level. While these regressions are somewhat underpowered, the point estimates suggest that farmers respond to increased groundwater costs by switching crops. 





